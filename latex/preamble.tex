% 使用fontspec扩展包,fontspec扩展包能直接使用系统的中文字体
\usepackage{fontspec}

% 使用xunicode扩展包,处理Unicode编码
\usepackage{xunicode}

% 使用listings扩展包,支持代码高亮等效果。
\usepackage{listings}

% 使用xeCJK扩展包,这样可分别设置中文和字母的字体,注意,这个宏包会被ctex宏包自动调用,因此无需手动调用
% \usepackage{xeCJK}

% 使用xcolor扩展包,这样能在文档中使用颜色
\usepackage{xcolor}

% 使用picinpar扩展包实现文字绕排
\usepackage{picinpar}

% 使用能生成超级链接的hyperref扩展包
\usepackage{hyperref}

% 使用自定义页眉页脚的fancyhdr扩展包
\usepackage{fancyhdr}

% 设置正文罗马字体族
\setmainfont{OpenSans-Light}

% 设置无衬线字体族
\setsansfont{Lato-Light}

% 设置打字机字体族
\setmonofont{Consolas}

\hypersetup{colorlinks,
        linkcolor=black,
        filecolor=black,
        urlcolor=blue,
        citecolor=black,}

% 设置中文自动换行
\XeTeXlinebreaklocale zh

%% 给予TeX断行一定自由度
\XeTeXlinebreakskip = 0pt plus 1pt

% 设置全局行间距
\linespread{1.5}

% 为代码扩展包listings设置全局性样式
\lstset{
	numbers=left,
	xleftmargin=2em,
	xrightmargin=2em,
	frame=shadowbox,
	basicstyle       = \small\ttfamily,
    stringstyle      = \ttfamily,
	tabsize          = 2,
	breaklines       = true,
	frame            = single,
	frameround       = tttt,
	showstringspaces = false,
	keywordstyle     = \color{black},
  % commentstyle     = \color{red},
	aboveskip        = 1em,
	fontadjust       = true,
}
      

% 载入页面设置包,并进行页面设置
\usepackage[
	a4paper,
	inner         = 1.5cm,
	outer         = 3cm,
	top           = 2cm,
	bottom        = 3cm,
	bindingoffset = 1cm
  ]{geometry}

% 载入页眉页脚包
\usepackage{fancyhdr}

% 设置页眉页脚格式
\pagestyle{fancy}
\fancyhf{}
\cfoot{\thepage}


\setlength{\headheight}{12pt}

% 载入pifont包,并设置脚注样式为带圈数字
\usepackage{pifont}
\renewcommand\thefootnote{\ding{\numexpr171+\value{footnote}}}

% 使用footmisc宏包,并设置脚注为每一页都重新编号
\usepackage[perpage]{footmisc}

% 设置参考文献的样式为方括号中的数字
\usepackage[super,square]{natbib}
\bibliographystyle{gbt7714-2005}

% 使用titlesec宏包,设置标题的字体、大小
\usepackage{titlesec}
\newfontfamily\hwzs{华文中宋}
\newcommand{\sectionfontsize}{\fontsize{18pt}{18pt}\selectfont}
\titleformat{\section}[hang]{\bfseries\hwzs\sectionfontsize}{\thesection}{1em}{}{}

% 方便表格的处理
\usepackage{multirow}
\usepackage{booktabs}

%方便数学公式的输入
\usepackage{mathtools}

%方便对插入图片的处理
\usepackage{graphics}

%对文章内容进行分栏
\usepackage{multicol}

% 方便数学字体(直立体小写希腊字母)的输入
\usepackage{upgreek}