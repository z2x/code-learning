\documentclass[12pt, a4paper,openany]{book}%book默认新的章节从奇数页开始,通常采用openany选项,使得新的章节可以从奇数页和偶数页开始
\usepackage{ctex}%使用中文字体宏包
\usepackage{amsfonts}
\usepackage{graphicx}%图片宏包
\usepackage{hyperref}%超链接宏包
\usepackage{fancyhdr}%页眉页脚宏包
\usepackage{listings}% 使用listings扩展包,支持代码高亮等效果
% 为代码扩展包listings设置全局性样式
\lstset{
	basicstyle = \ttfamily,
	language   = TeX,
	tabsize    = 2,
	breaklines = true,
	frame      = single
}
\title{\LaTeX{}学习手册}
\author{朱恩迪}
\date{}
\begin{document}
\maketitle
\tableofcontents %列出目录
\chapter{初学内容}
\section{牛刀小试}
Hello, world.						%转行使用回车键

你好,世界

\section{初入战场}
\subsection{改变字体看一下}
这个字显示为正体,					%字体显示方式的改变
这个字显示为{\bf 粗体}(boldface),
这个字显示为{\it 花体}(italic)\\
字体大小改变看看{\tiny\LaTeX{}},超大号字体显示{\huge\LaTeX{}}

\subsection{回味下数学公式}
$x + 2y = 3$,$x^{b+2}=8$,$x^2=8$  %数学符号的简单运用

$\lim_{\Delta x\to 0} \frac{f(x_0+\Delta x) - f(x_0)}{\Delta x}$,The anti-derivative of $1/x$ is $$ \int \frac{1}{x} \,dx = \ln x + C $$ where $C$ is an arbitrary constant.

\subsection{列举给你看看}
无序列举(itemize)
\begin{itemize}
\item Calculus
\item Linear Algebra
\item Basic Computer Concepts
\end{itemize}

有序列举(enumerate)
\begin{enumerate}
\item one
\item two
\item three
\end{enumerate}

提示列举(description)
\begin{description}
\item [Calculus] Integration and Differentiation
\item [Linear Algebra] Vector space and Basis
\item [Basic Computer Concepts] Programming Language
\end{description}

\subsection{上标与数学公式}
%上标在数学公式中的简单应用
For each $\epsilon > 0$, there exists a $\delta > 0$
such that, if $0 < |x - c| < \delta$, 
then $|f(x) - L| < \epsilon$

简单分式的使用方法
$n/2$=$(m+n)/n$

使用sqrt来编辑根号,,例如:
$\sqrt{x+y}$

\subsection{手写字体与黑板粗体字}
使用mathcal命令来输入手写字体,字母中不能输入小写字母,例如:
Choose $\mathcal{F}$ such that $\mathcal{F}(x) > 0$,$\mathcal{A}$

黑板粗體字 (Blackboard Bold)的使用方法:(1)使用amsfonts套件,并插入Bbb命令,例如:
$\Bbb{A}$,$\Bbb{B}$,$\Bbb{A}$

\subsection{向左走向右走}
箭头符号的使用实例:
向左走:$\Leftarrow$,向右走:$\Rightarrow$,左右徘徊:$\Longleftrightarrow$

\subsection{看看在正文中显示控制命令}
\TeX{}是一种优秀的电子排版系统。该系统由美国著名计算机学家高德纳(唐纳德·尔文·克努斯)发明。

\LaTeX{}是\TeX{}中的一种格式(format),是建立在\TeX{}基础上的宏语言,也就是说,
每一个\LaTeX{}命令实际上最后都会被转换解释成几个甚至上百个\TeX{}命令。
但是,普通用户可以无需知道这中间的复杂联系。\LaTeX{}根据人们排版文章的习惯,定义了许多命令和模板,通过这些命令和模板,让普通用户能快速得到漂亮的排版结果。

\subsection{如何设置章节}
注意:章节只有在文档类型为书籍时,才可正常使用。
使用 chapter来进行章节设置。

\subsection{脚注的设置}
使用footnote来进行脚注设置,例如:本书作者为朱恩迪\footnote{1993-?}

\subsection{插入图形}
使用宏包的插图功能,需要使用graphicx,然后就可以使用includegraphics命令插入图片:

\includegraphics[width=6cm]{desk.jpg}

很大的图片如果使用固定位置列出,不方便进行排版和分页,因此需要使用浮动体(float)来使插入的图片放在可以变动的相对位置环境中,例如:

\begin{figure}[ht]
\centering
\includegraphics[scale=0.3]{desk.jpg}
\caption{我的桌面壁纸其实是这样子}
\label{fig:desk}
\end{figure}

\subsection{插入超链接}
在导言区载入 hyperref 宏包,会自动建立目录、脚注和参考文献的内部超级链接。
如果还不会,\href{http://www.baidu.com}{百度}一下,你就知道。

\subsection{插入页眉页脚}
在\LaTeX{}中预设了四种页眉页脚的样式,分别是 empty、plain、headings 和
myheadings。若要使用用自定义页眉页脚, 一般都要使用宏包 fancyhdr。

\fancyhf{}
	\fancyhead[LEO]{\LaTeX{}学习手册}
	\pagestyle{fancy}
\subsection{原来两个反斜杠可以换行}
是不是真的呢?
真的换行了,不过没有缩进哟。

\chapter{\LaTeX{}绘制图表}
\section{表格环境的定义}
环境tabular和tabular是生成表格的基本工具,其定义(语法)为:


\chapter{\LaTeX{}插图指南}

\section{前言}
在\LaTeX{}可以可以使用几乎所有的图片格式,但eps是最早引入\LaTeX{}中的图像格式,因此对他的支持也是最好的。

要在\LaTeX{}中插入一幅图像,可以使用以下命令来调用宏包:
	\begin{lstlisting}
		\usepackage{graphicx}
	\end{lstlisting}

然后,在文件中使用以下命令来添加图片:
	\begin{lstlisting}
		\includegraphics{file.eps}
	\end{lstlisting}

还可以使用width和height选项来定义图像的宽度和高度:
	\begin{lstlisting}
		\includegraphics[width=4cm]{file.eps}
		\includegraphics[height=3cm]{file.eps}
	\end{lstlisting}

另外,使用angle命令来旋转图像:
	\begin{lstlisting}
		\includegraphics[angle=45]{file.eps}
	\end{lstlisting}

\section{graphicx相关选项解释}
	\begin{tabular}{ll}
		height & 图形的高度 \\
		totalheight & 图形的全部高度 \\
	\end{tabular}
\end{document}